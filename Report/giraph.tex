\section{Apache Giraph}
\label{sec:giraph}

Description of Apache Giraph\cite{ApacheGiraph}.

Apache Giraph (Apache Software Foundation 2012), is a popular open-source implementation of
Pregel. Giraph uses Map-only Hadoop jobs to schedule and coordinate the 
vertex-centric workers and uses HDFS for storing and accessing graph datasets. 
It is developed in Java and has a large com- munity of developers and 
users such as Facebook\cite{GiraphAtFacebook}. 
Giraph has a faster input loading time compared to Pregel 
because of using byte array for graph storage. On the other hand, this method 
is not efficient for graph mutations, which lead to decentralized edges when removing an edge. 
Giraph inherits the benefits and deficiencies of the Pregel vertex-centric
programming model. Its performance and scalability is algorithm and graph dependent, 
and works very fast, e.g., on stationary algorithms like PageRank but not as fast on 
traversal algorithms like single source shortest path (SSSP) (Roy 2014) andWCCs 
(Salihoglu and Widom 2014), particularly for graphs with a large diameter. 
However, the ease of use of this framework and the community support has made it a 
popular platform over which to develop other Pregel-like systems with feature enhancements 
to the vertex-centric concept.


Why pagerank?
Needs to be global, traversing the whole graph. Examples include page rank and connected
components, degree distribution.