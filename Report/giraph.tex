\section{Apache Giraph}
\label{sec:giraph}

Description of Apache Giraph\cite{ApacheGiraph}.

Apache Giraph (Apache Software Foundation 2012), is a popular open-source implementation of
Pregel. Giraph uses Map-only Hadoop jobs to schedule and coordinate the 
vertex-centric workers and uses HDFS for storing and accessing graph datasets. 
It is developed in Java and has a large com- munity of developers and 
users such as Facebook\cite{GiraphAtFacebook}. 
Giraph has a faster input loading time compared to Pregel 
because of using byte array for graph storage. On the other hand, this method 
is not efficient for graph mutations, which lead to decentralized edges when removing an edge. 
Giraph inherits the benefits and deficiencies of the Pregel vertex-centric
programming model. Its performance and scalability is algorithm and graph dependent, 
and works very fast, e.g., on stationary algorithms like PageRank but not as fast on 
traversal algorithms like single source shortest path (SSSP) (Roy 2014) andWCCs 
(Salihoglu and Widom 2014), particularly for graphs with a large diameter. 
However, the ease of use of this framework and the community support has made it a 
popular platform over which to develop other Pregel-like systems with feature enhancements 
to the vertex-centric concept.


Why pagerank?
Needs to be global, traversing the whole graph. Examples include page rank and connected
components, degree distribution.

\subsection{Setting up Giraph}
1. Look at all the settings file and decribe important settings 


\subsubsection{Challenges for Beginner Giraph Users}

A significant amount of the time spent on this project was spent setting up our 
cluster and installing Giraph and all of its dependencies. Even for a 
well-supported and widely used piece of software like Giraph, this was a time 
consuming challenge. To begin with, version mismatches of Java, HDFS, Ubuntu, 
and Giraph itself caused a number of difficult-to-debug errors. For example, 
installing the most recent version of Java caused a mysterious build error that 
took hours to track down a workaround for (the workaround being to downgrade 
Java). Additionally, once Giraph had been successfully built, a total of eight 
hours was spent tuning various settings before the example job provided in the 
documentation would run without error. This was due to a number of factors. 
First, configuration settings are scattered across many different files, making 
it challenging to identify the source of a problem and difficult to predict how 
the settings would interact with each other. Second, few facilities existed for 
fine-tuning resource allocation. Many jobs failed with ``out of memory'' 
exceptions because it was difficult to tune the amount of memory they would be 
allocated. Third, error messages are not gathered in a central location - the 
existence of numerous log files made tracking down errors time consuming. 
Finally, Giraph seems to be a highly sensitive and delicate system. 
Unfortunately, as a result it is fragile: the slightest misallocation or 
imbalance of a resource causes jobs to fail with little explanation.

Our conclusion was that Giraph is a tool that is best used for large, 
heavy-duty workloads. For simple graph processing, one machine and writing 
one's own graph processing code may be the more straightforward solution. It 
seems likely that Giraph is in fact optimized for such workloads, given the 
difficulty we experienced in trying to make Giraph use ``small'' input graphs 
(less than a million 
nodes).

\subsection{Effect of different partition schemes}